\documentclass[12pt, a4paper]{article}
\usepackage[utf8]{inputenc}
\usepackage{polski}
\usepackage{hyperref}
\usepackage{graphicx}
\usepackage{algorithm}
\usepackage{algpseudocode}
\usepackage{amsmath}
\usepackage{amsfonts}
\usepackage{float}
\usepackage{geometry}
\title{\textbf{Porównanie klasycznej ewolucji różnicowej DE/rand z odmianą przesuwającą punkt środkowy DE/mid}}
\author{Adam Stelmaszczyk}
\date{\today}
\setlength{\parindent}{0in}
\renewcommand\refname{Referencje}

\def\C{{\rm C}\,}
\def\E{{\rm E}\,}

\begin{document}
\maketitle

\begin{abstract}
W pracy porównano klasyczny algorytm ewolucji różnicowej DE/rand z jego odmianą DE/mid.
W części teoretycznej wyprowadzono parametry skalujące dla obydwu algorytmów. W części praktycznej
przeprowadzono szereg eksperymentów na 7 wybranych funkcjach z BBOB 2013 \cite{hansen}. DE/mid
okazał się lepszy na większości z nich.
\end{abstract}

\section{Część teoretyczna}

Klasyczna ewolucja różnicowa DE/rand oraz jej odmiana DE/mid różnią się jedynie operatorem mutacji.
W poniższych rozdziałach przedstawiono operatory mutacji dla DE/rand/k, DE/mid/k, DE/rand/$\infty$, DE/mid/$\infty$
oraz wyprowadzono wzory na współczynniki skalujące.

\subsection{Mutacja w DE/rand}

W DE/rand/1, mutant $i$-tego osobnika w populacji $P$~o~$n$~osobnikach powstaje w następujący sposób \cite{opara}:
\begin{equation} \label{eq:derand1}
u_i = P_{i_1} + F(P_{i_2} - P_{i_3})
\end{equation}

$i_1, i_2, i_3$ to indeksy wylosowane zgodnie z rozkładem jednostajnym ze zbioru \\ 
$\{0, 1, \dots, n-1\}$. Zatem $P_{i_1}, P_{i_2}, P_{i_3}$ to rozwiązania wylosowane zgodnie z rozkładem jednostajnym z populacji $P$.
$F\in\mathbb{R_+}$ to współczynnik skalujący, w tej pracy ustalony na 0,9. \\

DE/rand/1 można uogólnić na DE/rand/k, w którym to dodajemy
$k \in \mathbb{N}$ wektorów różnicowych:
\begin{equation} \label{eq:derand}
u_i' = P_{i_1} + F_k\sum\limits_{j=1}^k (P_{i_{2j}} - P_{i_{2j+1}})
\end{equation}

Podobnie jak poprzednio, $i_1, i_2, \dots i_{2k+1}$ to indeksy wylosowane zgodnie z rozkładem jednostajnym ze zbioru 
$\{0, 1, \dots, n-1\}$. Zatem $P_{i_1}, P_{i_2}, \dots, P_{2k+1}$ to rozwiązania wylosowane zgodnie z rozkładem 
jednostajnym z populacji $P$. $F_k\in\mathbb{R_+}$ to współczynnik skalujący dla DE/rand/k. 
Żeby macierz kowariancji populacji w DE/rand/k nie zmieniała się wraz z zmianą $k$, 
macierz kowariancji mutanta $u_i'$ musi być taka sama jak macierz kowariancji mutanta $u_i$.
Można to osiągnąć tak dobierając $F_k$, aby było spełnione równanie:
\begin{equation} \label{eq:kowariancje}
\C[u_i] = \C[u_i']
\end{equation}

Osobniki są liniowo niezależne od siebie, dlatego:
\begin{align*}
\C[u_i] \overset{(\ref{eq:derand1})}{=} \C[P_{i_1} + F(P_{i_2} - P_{i_3})] = \C[P_{i_1}] + F^2(\C[P_{i_2}] + \C[P_{i_3}])
\end{align*}

$\forall{i}\hspace{1mm}\C[P_i] = \C[P]$, ponieważ każdy osobnik ma taki sam rozkład prawdopodobieństwa. Zatem:
\begin{equation} \label{eq:macierz_kow_mutanta}
\C[u_i] = \C[P] + F^2(\C[P] + \C[P]) = \C[P](2F^2 + 1)
\end{equation}

Rozwijając prawą stronę równania (\ref{eq:kowariancje}):
\begin{align*}
\C[u_i'] \overset{(\ref{eq:derand})}{=} \C[P_{i_1} + F_k\sum\limits_{j=1}^k (P_{i_{2j}} - P_{i_{2j+1}})] 
= \C[P_{i_1}] + F_k^2\C[\sum\limits_{j=1}^k (P_{i_{2j}} - P_{i_{2j+1}})] \\
= \C[P_{i_1}] + F_k^2\C[\sum\limits_{j=2}^{2k+1} P_{i_{j}}] \\
= \C[P](2kF_k^2 + 1)
\end{align*}

Podstawiając do (\ref{eq:kowariancje}):
\begin{align*}
\C[P](2F^2 + 1) = \C[P](2kF_k^2 + 1)
\end{align*}

Zakładając, że $\C[P] \neq \textbf{0}$:
\begin{align*}
F^2 = kF_k^2
\end{align*}

Obie strony są nieujemne, więc ostatecznie:
\begin{align*}
F_k = \frac{F}{\sqrt{k}}
\end{align*}

\subsection{Mutacja w DE/mid}

W DE/mid/k mutant powstaje w następujący sposób:

\begin{equation} \label{eq:demid}
u_i'' = m + F_m\sum\limits_{j=1}^k (P_{i_{2j}} - P_{i_{2j+1}})
\end{equation}

Jedyną różnicą w porównaniu do DE/rand/k jest $m$, czyli punkt środkowy populacji:
\begin{equation} \label{eq:midpoint}
m = \frac{1}{n}\sum\limits_{j=1}^n P_j
\end{equation}

$F_m\in\mathbb{R_+}$ jest współczynnikiem skalującym dla DE/mid/k, analogicznym do $F$ dla DE/rand/1. 
Żeby macierz kowariancji populacji w DE/mid była taka sama jak w DE/rand/1, 
macierz kowariancji mutanta $u_i''$ musi być taka sama jak macierz kowariancji mutanta $u_i$.
Można to osiągnąć tak dobierając $F_m$, żeby było spełnione równanie:
\begin{equation} \label{eq:rownanie}
\C[u_i] = \C[u_i'']
\end{equation}

Rozwijając prawą stronę równania (\ref{eq:rownanie}):
\begin{align*}
\C[u_i''] \overset{(\ref{eq:demid})}{=} \C[m + F_m\sum\limits_{j=1}^k (P_{i_{2j}} - P_{i_{2j+1}})] \\
\overset{(\ref{eq:midpoint})}{=} \C[\frac{1}{n}\sum\limits_{j=1}^n P_j] + F_m^2\C[\sum\limits_{j=1}^k (P_{i_{2j}} - P_{i_{2j+1}})] 
= \frac{1}{n^2}n\C[P] + F_m^2\C[\sum\limits_{j=2}^{2k+1} P] = \C[P](2kF_m^2 + \frac{1}{n})
\end{align*}

Podstawiając do (\ref{eq:rownanie}):
\begin{align*}
\C[P](2F^2 + 1) = \C[P](2kF_m^2 + \frac{1}{n})
\end{align*}

Przy założeniu, że $\C[P] \neq \textbf{0}$:
\begin{align*}
2F^2 + 1 = 2kF_m^2 + \frac{1}{n} \\
F_m^2 = \frac{2F^2 + 1 - \frac{1}{n}}{2k}
\end{align*}

Obie strony są nieujemne, więc:
\begin{align} \label{eq:a}
F_m\ = \sqrt{\frac{2F^2 + 1 - \frac{1}{n}}{2k}}
\end{align}

Przyjmując $F=0,9$ z (\ref{eq:a}) wynika, że: \\
$F_m \approx 1,14$ dla $k=1$ i $n\to\infty$. \\

W DE/mid przesuwany jest punkt środkowy $m$ zamiast losowo wybranego osobnika $P_{i_1}$.
Punkt środkowy jest mniej zmienny, 
tzn. norma macierzy kowariancji punktu środkowego jest mniejsza niż norma macierzy kowariancji dowolnego osobnika.
$\lim_{n\to\infty} \C[m] = \textbf{0}$, natomiast $\C[P_{i_1}] = \C[P]$.
Dlatego DE/mid/k potrzebuje większego współczynnika skalującego niż DE/rand/k. \\

\subsection{Mutacja w DE/rand/$\infty$}

Zgodnie z centralnym twierdzeniem granicznym, $\frac{1}{{\sqrt{k}}}\sum\limits_{j=1}^k (P_{i_{2j}} - P_{i_{2j+1}})$ 
zbiega według rozkładu do $\mathcal{N}(0, \C[P])$ gdy $k \to \infty$. 
Dzięki temu, równanie mutanta DE/rand/$\infty$ można zapisać jako:
\begin{align*}
u_i = P_{i_1} + F_\infty \cdot v_\infty
\end{align*}

Gdzie $v_\infty \sim \mathcal{N}(0, \C[P])$. Wyznaczmy $F_\infty$.
\begin{align*}
\C[u_i] = \C[P_{i_1} + F_\infty \cdot v_\infty] \overset{(\ref{eq:macierz_kow_mutanta})}{=} \C[P](2F^2 + 1) \\
\C[P] + \C[F_\infty \cdot v_\infty] = \C[P](2F^2 + 1) \\
\C[F_\infty \cdot v_\infty] = 2F^2\C[P] \\
F_\infty^2 \C[P] = 2F^2\C[P] \\
F_\infty^2 = 2F^2 \\
F_\infty = \sqrt{2}F
\end{align*}

\subsection{Mutacja w DE/mid/$\infty$}

Równanie mutanta DE/mid/$\infty$ można zapisać podobnie:
\begin{align*}
u_i' = m + F_{\infty_m} \cdot v_\infty
\end{align*}

Wyznaczmy $F_{\infty_m}$.
\begin{align*}
\C[u_i'] = \C[m + F_{\infty_m} \cdot v_\infty] \overset{(\ref{eq:macierz_kow_mutanta})}{=} \C[P](2F^2 + 1) \\
\C[m] + C[F_{\infty_m} \cdot v_\infty] = \C[P](2F^2 + 1) \\
\frac{C[P]}{n} + F_{\infty_m}^2 C[P] = \C[P](2F^2 + 1) \\
F_{\infty_m} = \sqrt{2F^2 + 1 - \frac{1}{n}}
\end{align*}

\subsection{Podsumowanie}

Tabela \ref{table:parametry} podsumowuje znalezione współczynniki skalujące.

\begin{table}[H]
\centering
\begin{tabular}{ l | l }
algorytm         & współczynnik \\ \hline
DE/rand/k        & $\sqrt{\frac{2F^2}{2k}} = \frac{F}{\sqrt{k}}$ \\ 
DE/rand/$\infty$ & $\sqrt{2F^2} = \sqrt{2}F$ \\ \hline
DE/mid/k         & $\sqrt{\frac{2F^2 + 1 - \frac{1}{n}}{2k}}$ \\
DE/mid/$\infty$  & $\sqrt{2F^2 + 1 - \frac{1}{n}}$ \\
\end{tabular}
\caption{Współczynniki skalujące}
\label{table:parametry}
\end{table}

\section{Część praktyczna}

Przetestowano 6 algorytmów:
\begin{enumerate}
 \item DE/rand/1
 \item DE/rand/6
 \item DE/rand/$\infty$ 
 \item DE/mid/1 
 \item DE/mid/6 
 \item DE/mid/$\infty$ 
\end{enumerate}

Eksperymenty przeprowadzono na 7 funkcjach testowych o numerach 15, 16, 19, 20, 21, 22, 24 z BBOB 2013 \cite{finck}, zaimplementowanych w języku C.
Funkcje testowe są wywoływane z Javy, w której napisano algorytmy oraz procedurę testującą.
Liczba wymiarów $D \in \{20, 40, 80\}$. Maksymalna liczba wywołań funkcji oceny $FEs = 10^5D$. Rozmiar populacji $n = 10D$. 
Jeśli algorytm nie znajdował minimum, wówczas w jednym uruchomieniu, na jednej funkcji, generował $\frac{FEs}{n} = 10^4$ pokoleń.
Na każdej funkcji algorytm był niezależnie uruchamiany 15 razy, z każdego uruchomienia zapisywany był najlepszy wynik.
$F = 0.9$. Krzyżowanie wymieniające (DE/*/k/bin).
Prawdopodobieństwo krzyżowania $Cr = 0.9$. \\

W celu porównania, wykreślano dystrybuanty empiryczne najlepszych wyników z każdego uruchomienia 
dla obu algorytmów na jednej funkcji. Wykresy przedstawiono poniżej.
Najlepszym wynikiem była najmniejsza odległość funkcji oceny osobnika od minimum.
Algorytm, którego dystrybuanta na wykresie przebiegała powyżej pozostałych, otrzymywał 5 punktów. 
Za drugie miejsce algorytm dostawał 4 punkty, za trzecie 3, za czwarte 2, za piąte 1, za ostatnie 0. 
Jeśli dystrybuanty się przecinały, algorytmy zajmowały i-te miejsce ex aequo i dostawały punkty za i-te miejsce.
Przykładowo, na 16 funkcji w 20 wymiarach, DE/rand/6 i DE/mid/$\infty$ zajęły ex aequo drugie miejsce, dostając po 4 punkty.
Kolejny, DE/rand/$\infty$, zajął 3 miejsce i dostał 2 punkty.
Wyniki dla 20 wymiarów przedstawia tabela \ref{table:20d}, dla 40 wymiarów przedstawia tabela \ref{table:40d},
dla 80 wymiarów tabela \ref{table:80d}. \\ 

\begin{table}[H]
\centering
\begin{tabular}{ l | c | c | c | c | c | c | c | c}
                 & 15 & 16 & 19 & 20 & 21 & 22 & 24 & suma \\ \hline
DE/rand/1        & 1  & 0  & 0  & 1  & 1  & 5  & 0  & 8    \\ 
DE/rand/6        & 4  & 4  & 3  & 5  & 5  & 5  & 5  & 31   \\ 
DE/rand/$\infty$ & 2  & 2  & 2  & 4  & 5  & 5  & 1  & 21   \\ 
DE/mid/1         & 5  & 1  & 1  & 2  & 5  & 5  & 2  & 21   \\
DE/mid/6         & 0  & 5  & 5  & 0  & 0  & 0  & 5  & 15   \\ 
DE/mid/$\infty$  & 3  & 4  & 4  & 4  & 2  & 1  & 5  & 23   \\
\end{tabular}
\caption{Porównanie algorytmów w 20 wymiarze}
\label{table:20d}
\end{table}

\begin{table}[H]
\centering
\begin{tabular}{ l | c | c | c | c | c | c | c | c}
                 & 15 & 16 & 19 & 20 & 21 & 22 & 24 & suma \\ \hline
DE/rand/1        & 0  & 0  & 0  & 0  & 0  & 0  & 0  & 0    \\ 
DE/rand/6        & 5  & 4  & 3  & 4  & 5  & 5  & 5  & 31   \\ 
DE/rand/$\infty$ & 2  & 2  & 1  & 4  & 5  & 5  & 2  & 21   \\ 
DE/mid/1         & 5  & 1  & 2  & 2  & 2  & 2  & 1  & 15   \\
DE/mid/6         & 1  & 5  & 5  & 1  & 2  & 2  & 3  & 19   \\ 
DE/mid/$\infty$  & 3  & 4  & 4  & 5  & 3  & 3  & 5  & 27   \\
\end{tabular}
\caption{Porównanie algorytmów w 40 wymiarze}
\label{table:40d}
\end{table}

\begin{table}[H]
\centering
\begin{tabular}{ l | c | c | c | c | c | c | c | c}
                 & 15 & 16 & 19 & 20 & 21 & 22 & 24 & suma \\ \hline
DE/rand/1        & 0  & 0  & 0  & 0  & 2  & 2  & 2  & 6    \\ 
DE/rand/6        & 5  & 3  & 3  & 3  & 5  & 5  & 5  & 29   \\ 
DE/rand/$\infty$ & 1  & 2  & 1  & 1  & 2  & 2  & 2  & 11   \\ 
DE/mid/1         & 4  & 2  & 2  & 4  & 2  & 2  & 2  & 18   \\
DE/mid/6         & 2  & 5  & 5  & 2  & 3  & 3  & 3  & 23   \\ 
DE/mid/$\infty$  & 3  & 4  & 4  & 5  & 4  & 4  & 4  & 28   \\
\end{tabular}
\caption{Porównanie algorytmów w 80 wymiarze}
\label{table:80d}
\end{table}

\pagenumbering{gobble}

\begin{figure}
\centering
\includegraphics[scale=0.75]{../pngs/20/15.png} 
\includegraphics[scale=0.75]{../pngs/20/16.png}
\end{figure}

\begin{figure}
\centering
\includegraphics[scale=0.75]{../pngs/20/19.png} 
\includegraphics[scale=0.75]{../pngs/20/20.png}
\end{figure}

\begin{figure}
\centering
\includegraphics[scale=0.75]{../pngs/20/21.png} 
\includegraphics[scale=0.75]{../pngs/20/22.png}
\end{figure}

\begin{figure}
\centering
\includegraphics[scale=0.75]{../pngs/20/24.png} 
\includegraphics[scale=0.75]{../pngs/40/15.png} 
\end{figure}

\begin{figure}
\centering
\includegraphics[scale=0.75]{../pngs/40/16.png} 
\includegraphics[scale=0.75]{../pngs/40/19.png} 
\end{figure}

\begin{figure}
\centering
\includegraphics[scale=0.75]{../pngs/40/20.png} 
\includegraphics[scale=0.75]{../pngs/40/21.png} 
\end{figure}

\begin{figure}
\centering
\includegraphics[scale=0.75]{../pngs/40/22.png} 
\includegraphics[scale=0.75]{../pngs/40/24.png} 
\end{figure}

\begin{figure}
\centering
\includegraphics[scale=0.75]{../pngs/80/15.png} 
\includegraphics[scale=0.75]{../pngs/80/16.png} 
\end{figure}

\begin{figure}
\centering
\includegraphics[scale=0.75]{../pngs/80/19.png} 
\includegraphics[scale=0.75]{../pngs/80/20.png} 
\end{figure}

\begin{figure}
\centering
\includegraphics[scale=0.75]{../pngs/80/21.png} 
\includegraphics[scale=0.75]{../pngs/80/22.png} 
\end{figure}

\begin{figure}
\centering
\includegraphics[scale=0.75]{../pngs/80/24.png} 
\end{figure}

\bibliographystyle{plain}
\bibliography{references}

\end{document}